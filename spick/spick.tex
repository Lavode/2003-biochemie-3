\documentclass[a4paper,twocolumn,usegeometry,english,fontsize=6,DIV=16]{scrartcl}

% Localization options
\usepackage[english]{babel}
\usepackage{blindtext}
\usepackage[T1]{fontenc}
\usepackage[utf8]{inputenc}

% Enhanced verbatim sections. We're mainly interested in
% \verbatiminput though.
\usepackage{verbatim}

% PDF-compatible landscape mode.
% Makes PDF viewers show the page rotated by 90°.
\usepackage{pdflscape}

% Houston, we need to save space.

\usepackage[bottom=2cm,top=2cm]{geometry}

% No vertical space inbetween lists.
\usepackage{enumitem}
\setlist{nosep}

% Remove space between paragraphs.
\setlength{\parskip}{0pt}
\setlength{\parsep}{0pt}
\setlength{\headsep}{0pt}
\setlength{\topskip}{0pt}
\setlength{\topmargin}{0pt}
\setlength{\topsep}{0pt}
\setlength{\partopsep}{0pt}

% Lower space between lines.
% \linespread{0.8}

% Get rid of space above and below section headings
\RedeclareSectionCommand[beforeskip=0pt,afterskip=0pt]{section}
\RedeclareSectionCommand[beforeskip=0pt,afterskip=0pt]{subsection}
\RedeclareSectionCommand[beforeskip=0pt,afterskip=0pt]{subsubsection}

% Advanced tables
\usepackage{tabu}
\usepackage{longtable}
\usepackage{dcolumn}
\newcolumntype{d}[1]{D{.}{\cdot}{#1} }

% Fancy tablerules
\usepackage{booktabs}

% Graphics
\usepackage{graphicx}

% Current time
\usepackage[useregional=numeric]{datetime2}

% Float barriers.
% Automatically add a FloatBarrier to each \section
% \usepackage[section]{placeins}

\usepackage{layout}

% Math tools
\usepackage{mathtools}
% Math symbols
\usepackage{amsmath,amsfonts,amssymb}
\usepackage{amsthm}

% SI units
\usepackage{siunitx}
\DeclareSIUnit\Molar{\textsc{m}}
\DeclareSIUnit\mole{mol}
\DeclareSIUnit\rpm{rpm}
\DeclareSIUnit\cfu{cfu}

% Chemistry
\usepackage{mhchem}

% Subfigures & captions
\usepackage{subcaption}
\usepackage{wrapfig}

\DeclarePairedDelimiter\abs{\lvert}{\rvert}
% 
% Source code & highlighting
\usepackage{listings}

% Convenience commands
\newcommand{\mailsubject}{2003 - Biochemie 3}
\newcommand{\maillink}[1]{\href{mailto:#1?subject=\mailsubject}
                               {#1}}

% Should use this command wherever the print date is mentioned.
\newcommand{\printdate}{\today}

% Empty header/footer
\pagestyle{empty}
% Disable header/footer on title page too
\renewcommand*{\titlepagestyle}{empty}

\subject{1611 - Biochemie 3}

\title{Summary}

\author{Michael Senn \maillink{michael.senn@students.unibe.ch} - 16-126-880}

\date{\printdate}

% Needs to be the last command in the preamble, for one reason or
% another. 
\usepackage{hyperref}

\begin{document}

% Space, we need space
% \maketitle
% \vspace*{-20pt}
% \enlargethispage{40pt}

\section{Gene und Chromosome}

\subsection{DNA als Erbsubstanz und Struktur der DNA}

\begin{description}
	\item[Transkription] DNA -> RNA durch RNA Polymerase. In Bakterium in
		Cytoplasma, in Eukaryonten in Zellkern inklusive pre-processing
		(splicing, capping)
	\item[Translation] RNA -> Protein durch Ribosome. In Bakterium und
		Eukaryonten in Cytoplasma.
	\item[DNA \& RNA-bindende Proteine] Schützen, organisieren \&
		regulieren ua durch Konformationsänderung Aspekte des
		Metabolismus. Non-specific binden unabhängig von Sequenz durch
		zB elektrostatische Interaktionen mit Phosphat Gruppen, für zB
		Chromosom Kompaktierung. Specific erkennt spezifisch Sequenz
		durch H-Brücken mit BP, zB Regulation der Gen Expression.
		Kooperativität: Konformationsänderung einer Untereinheit führt
		zu Konfand anderer UE. Stark reguliert.
	\item[Nukleosid / Nukleotid] (Desoxy)ribose \& Stickstoffbase vs
		Nukleosid \& Phosphatgruppe
	\item[Chemische Transformation von Nukleinsäuren] Spontan (Mutationen)
		zB Deaminierung, enzymatisch reguliert zB Methylierung
	\item[Base stacking] Stapeln der hydrophoben BP führt zu weniger
		Kontakt mit Wasser, fördert Stabilität der DNA Helix.
	\item[DNA Schmelzpunkt Einfluss] GC führt zu höherer Temperatur
		(H-Bond), Ionenkonzentration stabilisiert Phosphat-Gruppen,
		Chaotrope Stoffe destabilisieren H-Bond, pH (de)protoniert
		Gruppen.
	\item[Hairpin / Cruciform / Mirror repeat] Hairpin falls
		Schlaufenbildung aufgrund eines 'inverted repeats' einer
		single-stranded DNA. Cruciform falls Doppelte Schlaufe wegen
		double-stranded. Mirror repeat falls Sequenz mit 'Spiegelachse'
		in der Mitte.
	\item[A / B / Z Form] B Form am stabilsten unter physiologischen
		Bedingungen, \SI{3.4}{\angstrom} per BP, \SI{6}{\degree} BP
		tilt. A in Abwesenheit von Wasser, \SI{2.6}{\angstrom} per BP,
		\SI{20}{\degree} BP tilt. Z Form linksgängig,
		\SI{3.7}{\angstrom} per BP, \SI{7}{\degree} BP tilt, bildet
		sich unter Hochsalzbedingungen mit alternierenden
		Pyrmidin-Purin Sequenzen.
	\item[Supercoiling] Verdichtung zirkulärer DNA Moleküle durch
		verdrehen. L (linking Number) wie oft ein Strang um anderen
		gewunden. Änderung bedingt DNA Schnitt. Negative/positiver
		supercoil bei under/overwinding. Topoisomerasen verändern
		linking number. Typ 1 ohne ATP schneidet ein Strang, Typ 2 mit
		ATP schneidet beide. NB: Eukaryontische Topoisomerasen können
		nur positive supercoil aufheben, keine negative supercoil
		induzieren!
	\item[Nukleotiden] Purin: Adenine, Guanine, Pyrmidin: Uracil, Thymine,
		Cytosine.
\end{description}

\subsection{DNA Schäden \& Reparatur}

\begin{description}
	\item[DNA Schäden] Hydrolyse der Ribose-Basen Bindung (häufig) oder
		Phosphodiesterbindung (selten). Methylierung verschiedener
		Stickstoffe der Basen (mittel), Oxidierung der Basen oder
		Zucker (selten - mittel)
	\item[Mutation] Schaden der vor Replikation nicht behoben wurde. Treibende Kraft der Evolution, vererbbar.
	\item[Ames Test] Inkubation von Bakterien ohne His-Synthese auf Platten
		ohne His. Nur jene die durch Mutationen (spontan oder durch
		Stoff induziert) His syntethisieren können wachsen. Führt zu
		stärkerem Wachstum falls mit mutagenen Stoff in Filterpapier in
		Mitte der Platte behandelt, exklusive 'clear zone' in jenem
		Bereich bei welchem Stoff toxisch / zu viele Mutationen.
	\item[Substitutionsmutation / Punktmutation] Austasch eines BP gegen
		anderes. (Transition: Purin gg Purin / Pyrmidin gg Pyrmidin,
		Transversion: Anders). Aufgrund zweisträngiger DNA sind Point
		mutations nach 1 Generation noch erkennbar, nach 2ter nicht
		mehr.
	\item[Indel] Insertion / Deletion
	\item[Silent / neutral mutation] Ohne Auswirkung auf Genfunktion / AS-Sequenz
	\item[Single nucleotide polymorphism SNP] Kleine Mutationen in Genen
		von Individuuen gleicher Spezies.
	\item[Desaminierung] Ca 100 / Zelle / Tag. Erkennbar \& reparierbar da
		immer Degradation zu BP das nicht in DNA vorkommt.
	\item[Depurinierung] \si{10E-5} / Zelle / Tag. Verlust der Base. Führt
		zu -1 del nach Replikation.
	\item[Oxidative Modifikationen] durch zB ROS. Kann dazu führen dass
		Base mit neuer Base paaren kann, bei falscher Reparatur oder
		bei Replikation damit zu Point Mutation.
	\item[Basenmethylierung] durch alkylierende Substanzen. Kann dazu
		führen dass neue Basen-Paarungen, kann still sein, oder alles
		dazwischen.
	\item[Pyrimidindimere] Durch UV Strahlung. Basen-Basen Bindung führt zu
		Knick in DNA, blockiert Polymerase.
	\item[Rückgratbrucht durch ionisierende Strahlung]
	\item[Indirekte Reparatur] Betroffener Bereich abgebaut und neu
		synthetisiert.
	\item[Direkte Reparatur] Schaden repariert, modifizerte Base
		wiederhergestellt, Monomer gebrochen etc. Bsp Reparatur von
		Thymindimeren durch Photolyase, oder Transfer der Methyl-Gruppe
		alkylierter Basen auf `Opfer'-Molekül durch Methyltransferase.
	\item[Mismatch Reparatur MMR] Erkennt Template-Strang aufgrund
		Methylierung da neuer Strang nicht sofort methyliert. Damit MMR
		auf nicht-methylierten Strang/Strängen. Bildung Loop mit
		Mismatch darin zwischen zwei Methylierten Stellen, ausschneiden
		des Bereichs des Template Stranges, Replikation.
	\item[Base excision repair BER] Erkennen aller Basen die keine
		H-Brücken mehr bilden (Aufgrund zB Depurinierung oder
		Deaminierung). Ausschneiden des leeren Rückgrates, Replikation.
	\item[Nucleotide excision repair NER] Bei zB Pyrimidindimeren
		Ausschneiden eines n-mers des beschädigten Stranges,
		Replikation.
	\item[Doppelstrangbrüche] Während und kurz nach Replikation: Reparatur
		durch Homologe Rekombination (Ausrichtung an Schwesterstrang,
		dan erneute Replikation). Ansonsten: Nonhomologous end joining,
		Abschneiden des Überhanges, dann zusammenfügen.
	\item[SOS Response] DNA Schaden -> Inaktiviert LexA -> Expression SOS
		Gene (Mehr und weitere DNA Reparatur Gene, keine Zellteilung).
		Inaktivierung LexA durch RecA, welches a) DNA Schäden bindet
		und b) in gebundenem Zustand LexA spaltet.
\end{description}

\subsection{Chromatinstruktur}
\begin{description}
	\item[Chromatin] Komplex aus DNA um Histonen (= Nucleosom) und anderen
		Proteinen, Erlaubt Konzentration der DNA bis zu Faktor 700.
		Histone sehr basisch (Lys, Arg) / positiv, stark konserviert.
		Euchromatin locker, Heterochromatin gewunden (Histone
		kondensiert).  Acetylierung führt zu Öffnung, Methylierung zu
		Kondensation.
	\item[Histon] Protein Komplex, 8 UE, H2A, H2B, H3, H4, 146bp DNA.
		Helferproteine für Assembly des Oktamers. Interaktion mit DNA
		hauptsächlich mit Rückgrat und Minor Groove, damit
		\SI{75}{\percent} der DNA Oberfläche Zugänglich für zB
		Regulation. H1 `hält' DNA um Histon fest. Mit H1 200 bp.
	\item[Histonenden] Unstrukturiert / loose. Tragen zur höheren
		Organisation (zB \SI{30}{\nm} Faser) bei.
	\item[Unterwundene eukaryontische DNA] Binden DNA an Histon Core
		induziert negative Supercoil, damit positiven Supercoil in der
		Nähe. Dieser durch Topoisomerase aufgehoben, damit netto
		unterwunden.
	\item[Chromatin Immunoprecipitation (ChIP)] Isolierung Chromatin aus
		Zelle, Schneiden mit Nuclease (ca 500bp Fragmente) Bindung und
		Isolierung spezifischer Histon-Modifikationen mit Antikörpern,
		Extrahierung \& Sequenzierung DNA. Dadurch Erkenntniss in
		welchem DNA Bereich welche Histon-Modifikationen vorkommen.
	\item[SMC Proteine] Cohesin verhindert Dissoziation von
		Schwester-Chromatiden, Condensin helfen Chromatide kondensiert
		zu halten. Cohesin  wird in Anaphase entfernt damit Aufteilung
		auf Tochterzellen, nach Chromatinseparation dissazoziiert
		Condensin damit dekondensation der DNA.
\end{description}

\subsection{DNA Replikation}
\begin{description}
	\item[Semi-konservativ] Bei Replikation werden Stränge des
		Doppelstranges getrennt und b) bleiben Stränge intern als
		Einheit.
	\item[5' -> 3'] Des neuen Stranges, dh 3' -> 5' des Template Stranges.
		Grund: Neue Phosphat-Gruppe wird an alte Hydroxy-Gruppe
		angehängt. Benötigt RNA Primer.
	\item[Okazaki Fragmente] 1k-2k nt in Prokaryonten, 150-200 nt in
		Eukaryonten.
	\item[Replikation] Bidirektional ab ORI, Replication Fork bei Trennung
		des Doppelstranges. Benötigt RNA Primer und \ce{Mg2+} Ionen.
	\item[DNA Polymerase Fehlerrate] DNA Polymerasen haben tiefe Fehlerrate
		von \SIrange{10E-9}{10E-10} aufgrund a) Geometrie der
		Basenpaarung b) 3' -> 5' Exonukleasenaktivität (entfernen
		falscher nt) c) DNA Reparatur.
	\item[DNA Polymerasen in E. Coli] Pol3 (9 Polypeptide, 22 UE)
		Hauptenzym der Replikation. Beta-Clamp hält Polymerase auf DNA
		fest. Pol1 mit 5'->3' Exonukleasenaktivität für ua DNA
		Reparatur und Ersetzen des RNA Primers des Lagging Strand.
	\item[Replisom] Hilfsproteine zusätzlich zu Pol3 für Replikation, zB
		RNA Primase (Primer), Helicase (Auftrennen), Topoisomerase, SSB
		(Verhindern der Rehybridisierung), DNA Ligasen
	\item[Polymerasen-Komplex] Da zwei Stränge gleichzeitig repliziert zwei
		Polymerasen in Komplex, und Schlaufenbildung damit beide 5' ->
		3' Replikation und Polymerasen gegenläufig laufen können.
	% Todo Bild DNA Replikation, C4, pp30++
	\item[DNA Ligase] Aktiviert 5' durch Adenylierung (Eukaryonten ATP,
		Bakterien NAD+), dann Ausbildung Phosphodiesterbindung.
	\item[Termination der Replikaton] Richtungs-spezifische
		Terminuns-Regionen halten Replikationsgabel an. Topoisomerase
		entwindet die zwei Stränge.
	\item[Replikation in Eukaryonten] Verknüpft mit Zellzyklus (ua
		Chromatin), Replikation parallel an mehreren AT-reichen
		Stellen, ca 50 nt/s (Vgl mit 1000 nt/s in Bakterien). Primer
		hat RNA (Primase) und DNA (Pol alpha) Komponenten. Primer
		entfernt durch RNAse (RNA) und Flap endonuclease (DNA), Loch
		aufgefüllt durch Pol delta, geschlossen durch Ligase.
	\item[Telomere] G-reiche repetitive Sequenzen an Ende der Chromosomen,
		verhindert Informationsverlust in Zellen der Keimbahn (nicht
		aber somatische) durch lagging Strand. Wird jeweils wieder
		`aufgefüllt' durch Telomerase (451nt RNA Template und Reverse
		Transkriptase).  Telomerase verlängert einen Strand, zweite
		durch Pol. Verdau des Primers führt zu 3' Überhang, geschützt
		durch Proteine.
\end{description}

\subsection{DNA Rekombination}

\begin{description}
	\item[Genetische Rekombination] Austausch von genetischer Information
		innerhalb oder zwischen Chromosomen. Rolle in DNA Reparatur,
		Regulation, genetische Vielfalt, Gen Transfer.
	\item[Rekombinationstypen] Homolog: Austausch zwischen homologen (=
		ähnlichen) DNA Stücken. Site-specific: Rekombination an kurzen
		spezifischen Sequenzen. Mobile Transposons: Autonom
		replizierende und wandernde DNA Abschnitte.
	\item[Reparatur Doppelstrangbruch] Falls Schwester-Chromatid verfügbar:
		Ausrichtung eines Stranges des Schwesterchromatids an
		gebrochenem Strang. Schneiden der Holiday Struktur (X). Dann
		reguläre DNA Replikation (beide Chromatiden nun
		einzel-strängig). RecBCD Helikase/Nuklease bindet, baut Strang
		ab bis Chi-Sequenz, ab dort wie beschrieben Reparatur. RecA
		bindet an RecBCD (langsam) und ab dort an Einzelstrang
		(schnell), verhindert Abbau durch zB Nukleasen.
	\item[Homologe Rekombination währed Meiose] Doppelstrangbruch, Abbau 5'
		führt zu Überhang, Paarung des 3' Stranges mit Komplementärem
		Strang des Schwesterchromatid, Replikation, Auflösen Holiday
		Junction.
	\item[Site-specific Recombination] Einfügen und Ausschneiden bestimmter
		Sequenzen (zB Bakteriophagen), Änderungen der Genexpression.
		Recombinase erkennt Sequenzen, bricht DNA und rekombiniert.
		Zwei Familien mit gleicher Funktion, Integrase nutzt Tyrosin,
		Invertase/Resolvase nutzt Serin. Site-specific recombination
		führt zu Inversion, oder Insertion/Deletion.
	\item[Antikörper Diversität Immunoglobulingene] Konstante und variable
		funktionale Region. Durch site-specific Recombination werden
		viele verschiedene (eine je Zelle) Antikörper exprimiert.
		Sobald einer ein Antigen binden kann, wird jene Zelle
		repliziert - Vervielfachung des bindenden Antikörper.
	\item[Transposition mobiler genetischer Elemente] Transposons wandern
		durch Aktivität der Transposase. Autonom (encodieren eigene
		Transposase) vs non-autonom. Nach cut \& paste / copy \& paste
		site-specific Rekombination.
	\item[Retrotransposon] Transposons welche Zwischenschritt als RNA haben
		(analog Retroviren). LTR Retrotransposons encodieren reverse
		Transkriptase. LINE encodiert reverse Transkriptase aber ohne
		LTR (long terminal repeats). SINE ohne reverse Transkriptase,
		ohne LTR.
	\item[Transposons in Organismen] Mensch ca \SI{43}{\percent} des
		Genoms, aber bis auf einige LINEs alles inaktiv. Maus viele
		aktive Retro-Elemente, verantwortlich für \SI{10}{\percent} der
		Mutationen (vgl Mensch \SI{0.2}{\percent}.
	\item[Reverse Transkriptase Aktivität] DNA Polymerase: 3' OH Primer,
		RNA oder DNA als Template. \ce{Mg2+} katalysiert. Hohe
		Fehlerrate da kein Proof-reading. Ribonuklease: RNAse H
		Aktivität. Inhibitor: Nukleosid-Analog inhibieren cDNA, sowie
		allgemeine Inhibition von RT.
\end{description}

\section{Transkription - von der DNA zur RNA}

\subsection{RNA Polymerasen: Struktur und Funktion}

\begin{description}
	\item[Transkription] RNA Transkript ist an template Strang gebunden,
		und damit Kopie des coding / non-template Strangs. 
	\item[Bakterielle RNA Polymerase] Mehrere Untereinheiten (anders als
		bakterielle DNA Polymerase), keine Exonuklease Aktivität. Sechs
		untereinheiten, eine davon ($\sigma$) spezifiziert DNA
		Bindungsstelle / Ausrichtung.
	\item[$\alpha$-Amanitin] Inhibiert RNA Pol 2 (stark), 3 (schwach), 1
		(minimal)
	\item[Eurkaryontische RNA Polymerasen] Pol 1: 5.8S, 18S, 28s rRNA Gene
		(RNA Vorläufer). Pol 2: Protein-codierende Gene, snoRNA, snRNA,
		miRNA. Pol 3: tRNA, 5S rRNA, snRNA. Mehr Untereinheiten als
		bakterielle RNA Pol.
\end{description}

\subsection{Prokaryontische und Eukaryontische Transkription inklusive Steuersequenzen}

\begin{description}
	\item[Promotoren] -35 und -10 (Pribnow) AT-Reiche Sequenzen. Besonders
		stark exprimierte Gene noch -60 bis -40 konservierte `UP
		Element" Sequenz.
	\item[Initiations Stelle] $\Sigma$ Faktor bindet, rekrutiert Rest der
		Polymerase. Nach Promoter disasoziiert $\sigma$ und wird durch
		NusA ersetzt.
	\item[Terminationsstelle] Pol dissasoziiert. $\rho$-abhängig:
		Polymerase hält an Terminationsstelle an, $\rho$ Faktor bindet
		mRNA, migriert zu Pol unter ATP Verbrauch, löst Pol ab.
		$\rho$-unabhängig: GC-reicher Hairpin der mRNA stoppt RNAP,
		führt zu Ablösung der mRNA und Rehybridisierung der DNA.
	\item[Eurkarontische RNAP Promotoren] Pol1 ein Promoter,
		Spezies-spezifisch. Pol3 internal Promotoren (+40 bis +80) vs
		upstream (ähnlich RNA Pol 2)
	\item[RNAP 2 Promotoren] Länger als prokaryontische Promotoren.
		Housekeeping genes of CpG Inseln upstream (GC reich).
		Methylierung jener Cytosine erlaubt Inhibition der
		Promotoraktivität. Konservierte core promotoer Region mit  BRE,
		TATA, Inr (Initiator), Downstream core promoter.
	\item[Enhancer Elemente] Bindung durch regulatorische Proteine
		upstream. Interaktion mit Mediator, jener mit Pol.
	\item[Transkriptionsinitiation RNAP2] Bindung TBP und TF. Rekrutierung
		Pol2, Phosphorylierung der Pol an CTD (c-terminal-domian)
		startet Transkription, TF helfen Elongation, Dephosphorolation
		der CTD katalysiert durch Terminationsfaktor führt zu Auflösung
		der Pol2.
	\item[TBP] Sattelstruktur, interagiert mit minor groove (unüblich),
		biegt DNA. Universeller TF, bindet auch TATA-lose RNAP2 und
		RNAP1/3 Promoter.
	\item[RNAP2 TF] TBP wie oben. TF2A stabilisiert TF2B und TBP. TF2B
		bindet an TBP, rekrutiert Pol2-TF2F Komplex. TF2E rekrutiert
		TF2H, ATP-ase und Helicase Aktivität. TF2F bindet und richtet
		aus Pol2. TF2H entwindet DNA an Promoter (Helicase),
		Phosphoryliert CTD, rekrutiert weitere Proteine. Zusammen
		Äquivalenz zu $\sigma$ Einheit der Prokaryonten.
	\item[RNA2P EF] Erhöhen Aktivität der Polymerase.
\end{description}

\subsection{Transkriptionsregulation in Prokaryonten}

\begin{description}
	\item[Cis und trans-Faktoren] Cis: Sequenzen auf gleichem DNA Molekül
		zB Operator. Trans: Proteine / RNA welche sequenzspezifisch an
		Cis binden.
	\item[Repressor] Bindet an operator, blockiert Promoter, damit keine
		Bindung der Pol. Durch Signalmoleküle aktiviert oder
		inaktiviert.
	\item[Aktivator] Bindet an aktivator-binding-site, interagiert mit Pol
		und Promoter.  Durch Signalmoleküle aktiviert oder inaktiviert.
	\item[Specifity Factor] Dirigiert RNA Pol zu bestimmten Promoter.
	\item[Genregulation in Prokaryonten] Hauptsächlich auf Ebene der
		Transkription. Operon ist Sammlung von Genen hinter selbem
		Promoter.
	\item[Helix turn helix] Zwei $\alpha$ helices verbunden mit $\beta$
		Turn. Eine der Helices ist DNA-recognition Helix, bindet major
		groove. Häufig in Prokaryonten, seltener in Eukaryonten.
	\item[\ce{Zn2+} Finger] Längliche Loop, mit durch Cys/His koordiniertem
		\ce{Zn2+} Atom. Häufig in Eukaryonten, selten in Prokaryonten.
		Meist mehre Finger, einzelne Interaktion schwach.
	\item[Lac Regulation] Repressor blockiert Pol, ausser durch
		Lactose/IPTG gebunden. cAMP/CAP bindet upstream von -35 und
		stimuliert Pol2 Bindung ausser durch Glucose gebunden.
	\item[Trp Regulation] Repressor blockiert ausser durch Trp gebunden.
		Trp bindet weiter an Protein (dh NACH Ribosom) und führt zu
		Attenuation (Abbruch der Transkription weiter downstream durch
		Loop) falls Trp Konzentration hoch.
\end{description}

\subsection{Transkriptionsregulation in Eukaryonten}

\begin{description}
	\item[Fundamentaler Unterschied zu Eukaryonten] Aufgrund
		Chromatinstruktur nur Transkription wenn aktiviert. Bei
		Eukaryonten auch schwach Transkription falls nicht aktiviert.
	\item[Chromatin] Euchromatin locker, Hyperacetyierung. Heterochromatin
		kompakt, Methylierung.
	\item[HAT] Histon Acetyl Transferase. Acetyl-Coenzym A als Donor. Teil
		von Histon-Lockerungskomplexen. Acetylierung von Lys.
	\item[HMT] Histon Methyl Transferase. S-Adenosyl-Methionin (SAM) als
		Donor, Transfer auf Lys oder Arg.
	\item[Chromatin remodeling] Einfügen / entfernen von Histonen (ATP
		Verbrauch). Verschieben der DNA auf Histone (ATP Verbrauch).
		HAT/HMT.
	\item[Transkriptionsaktivatoren] Interagieren mit DNA und
		regulatorischen Proteinen (zB Steroidrezeptoren)
	\item[Architektonische Regulatoren] Herausloopen der DNA.
	\item[Co-Aktivatoren] Vermitteln zwischen RNAP und
		Transkriptionsaktivatoren (zB Mediator).
	\item[Bsp Steroid-Rezeptoren] Steroid diffundiert in Zelle, bindet an
		Rezeptor, Bildung eines Dimers zweier Rezeptoren, diffundiert
		in Zellkern, bindet HRE Region der DNA, bindet Coaktivator
		welcher RNA Pol bindet.  Aufbau: Schwach konservierte
		Aktivierungsdomäne, stark konservierte DNA Bindungsdomäne
		(\ce{Zn2+} Finger), spezifische Hormon-Bindungsdomäne.
\end{description}

\subsection{RNA-abhängige Synthese von RNA \& DNA}

\begin{description}
	\item[Reverse Transkriptase] Transkribiert RNA zu DNA. Bei Viren häufig
		mit RNAse H (Ribonuclease) Aktivität welche RNA des RNA/DNA
		Hybrids abbaut. Polymersation ebenfalls 5'->3'. DNA oder RNA
		als Template.
	\item[AZT] Thymidin-Analog. HIV-RT hat höhere Affinität für AZT als
		Thymidin, zelluläre DNA Pol tiefere. Führt zu Einbau von AZT
		nur durch RTs, dadurch Kettenabbruch.
\end{description}


\section{Post-transkriptionelle RNA Prozessierung}

\subsection{Eukaryontische mRNA: Capping, Splicing, Polyadenylierung}

\begin{description}
	\item[RNAP] Ribonukleotid-Protein-Partikel. RNA in Proteine gehüllt. In Zelle nie freie RNA.
	\item[Capping] Anhängen eines 5' 7-Methylguanosin Cap an Pol2
		Transkripte via 5'-5' Triphosphat.
	\item[RNA Splicing] Spliceosomal: Wie oben. Gruppe 1: Guanin
		Selbstangriff. Gruppe 2: RNA katalysiert in Pflanzen. pre-tRNA:
		Protein-katalysiert, weder self-splicend noch Introns.
	\item[Splicing] Ausschneiden der Introns. Exons nicht zwingend
		codierend, kann auch UTR sein (vor Start / nach End Codon).
		\SI{30}{\percent} des menschlichen Genoms Exon oder Intron,
		\SI{2}{\percent} der BP codierend. Splicing durch 3' splice
		site, 5' splice site, branch point (in Mitte). Branch point
		greift an und spaltet 5'. Gespaltene 5' (Exon-seitig) greift an
		und spaltet 3' (Exon-seitig). Intron löst sich.
	\item[Spliceosom] Proteinkomplex aus 150-300 Proteinen, 5 RNAs, der für
		jedes Intron auf- und abgebaut wird. Enthält U-snRNPs U1-U6.
		Potentiell Riboyzm, aber Proteine für Faltung essentiell, damit
		nicht in Isolation studierbar. Benötigt ATP für Zusammenfügen.
	\item[U-snRNPs] U-rich small nuclear ribonucleoprotein particles.
		100-200nts, 7 Sm Proteine, plus snRNP spezifische Proteine.
	\item[Exon junction complex] Proteinkomplex der 20-24 nt 5' der Exon
		Grenze bindet, und damit ehemaliges Intron markiert. Erlaubt
		Einfluss von Introns auf Genexpression.
	\item[Gruppe 1 self-splicing] Initiiert durch Guanin. 3' splicing site
		(Exon-seitig) greift 5' Splicing site an. Intron löst sich,
		bildet Loop. Kein ATP.
	\item[Gruppe 2 self-splicing] Nukleophil ist 2' OH eines Adenosins im
		Intron. Potentiell Vorläufer des Spliceosoms. Kein ATP.
	\item[Vorteile von Splicing] Mehrere Varianten eines Genes, Schutz vor
		Transposons / allgemeinen Mutationen.
	\item[Alternatives splicing] Allgemein: Gemeinsame pre-mRNA aus der
		verschiedene mRNA gemacht wird. Alternative exon:
		Auslassen/Einfügen eines bestimmten Exons. Potentiell nur
		Einseitig (5' Site vs 3' Site). Intron Retention: Fehlerhaftes
		belassen des Introns.  häufig verfrühter Translations-Stop da
		Stop-Codon. Mutually exclusive alternative exon: Eines von zwei
		Exons.
	\item[Alternative promoter and first exon] Transkriptions Start-Stelle
		verändert, Regulation auf Transkriptions-Ebene, damit kein
		alternatives splicing da verschiedene pre-mRNA.
	\item[ESE, ESS] ESE splicing enhancer, bindet Regulator welcher
		Spliceosom rekrutiert. ESS splicing silencer, verdeckt /
		blockiert splicing site. SR-Familie Beispiel von
		Splicing-Regulatoren. E.. exonic, I... intronic.
	\item[Somatische Sex Determination in Drosophilia] Kaskade. Sex-Lethal
		Protein (SXL) reguliert Splicing eigener mRNA. Bei weiblicher
		Version blockiert SXL Splicing Site, Verwendung späterer
		(End-)Site. Bei männlich frühere Site verwendet, enthält
		weiteres Exon welches Stop-Codon enthält.
	\item[Recruitment / Kinetic Splicing] Recruitement:
		Transkriptionskomplex rekrutiert Splicing Faktoren. Kinetic:
		Verwendung der stärksten erreichbaren Splicing Site, damit
		durch Transkriptionsgeschwindigkeit reguliert.
	\item[Polyadenylierung] Poly-A Schwanz an 3' Ende. Fördert Export ins
		Zytoplasma, verhindert Exonukleolytischer Verdau, Fördert teils
		Translation. Ausgelöst durch stark konserviertes PolyA-Signal
		kurz vor Schnittstelle. Zuerst Schnitt dann Polyadenylierung.
		Verwendet grossen Proteinkomplex welcher mit CTD der RPol2
		assoziiert.
	\item[CPSF-73] unspezifische Endonuklease die bei Polyadenylierung an
		zugänglicher Stelle schneidet.
	\item[Regulation der PolyA] Bsp B-Zellen Reifung. Inaktive Zellen:
		CstF-64 tief, bindung der distalen PolyA Stelle mit starker DSE
		(Element Downstream der PolyA Stelle). Aktive Zellen: CstF-64
		hoch, zusätzliche Bindung der proximalen PolyA Signal mit
		schwachem DSE.
	\item[Termination der RPol2] Alosterisches Modell:
		Konformationsänderung an Terminationssignal führt zu
		Termination. Torpedo Modell: Nach endonukleolytischem Schnitt
		entsteht 3' Ende mit freiem 5' welches schnell abgebaut wird,
		RPol2 fällt dadurch ab.
	\item[3' Prozessierung der Histon-mRNA] Ohne Introns, 3' Ende durch
		endonukleolytischen Schnitt ohne PolyA. Verwendet HLF
		Polyakomplex, ebenfalls CPSF-73 als Endonuclease.
\end{description}

\section{Protein Synthese}
Test

\end{document}
