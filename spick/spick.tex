\documentclass[a4paper,twocolumn,usegeometry,english,fontsize=6,DIV=16]{scrartcl}

% Localization options
\usepackage[english]{babel}
\usepackage{blindtext}
\usepackage[T1]{fontenc}
\usepackage[utf8]{inputenc}

% Enhanced verbatim sections. We're mainly interested in
% \verbatiminput though.
\usepackage{verbatim}

% PDF-compatible landscape mode.
% Makes PDF viewers show the page rotated by 90°.
\usepackage{pdflscape}

% Houston, we need to save space.

\usepackage[bottom=2cm,top=2cm]{geometry}

% No vertical space inbetween lists.
\usepackage{enumitem}
\setlist{nosep}

% Remove space between paragraphs.
\setlength{\parskip}{0pt}
\setlength{\parsep}{0pt}
\setlength{\headsep}{0pt}
\setlength{\topskip}{0pt}
\setlength{\topmargin}{0pt}
\setlength{\topsep}{0pt}
\setlength{\partopsep}{0pt}

% Lower space between lines.
% \linespread{0.8}

% Get rid of space above and below section headings
\usepackage[compact]{titlesec}
\titlespacing{\section}{0pt}{*0}{*0}
\titlespacing{\subsection}{0pt}{*0}{*0}
\titlespacing{\subsubsection}{0pt}{*0}{*0}

% Advanced tables
\usepackage{tabu}
\usepackage{longtable}
\usepackage{dcolumn}
\newcolumntype{d}[1]{D{.}{\cdot}{#1} }

% Fancy tablerules
\usepackage{booktabs}

% Graphics
\usepackage{graphicx}

% Current time
\usepackage[useregional=numeric]{datetime2}

% Float barriers.
% Automatically add a FloatBarrier to each \section
% \usepackage[section]{placeins}

\usepackage{layout}

% Math tools
\usepackage{mathtools}
% Math symbols
\usepackage{amsmath,amsfonts,amssymb}
\usepackage{amsthm}

% SI units
\usepackage{siunitx}
\DeclareSIUnit\Molar{\textsc{m}}
\DeclareSIUnit\mole{mol}
\DeclareSIUnit\rpm{rpm}
\DeclareSIUnit\cfu{cfu}

% Chemistry
\usepackage{mhchem}

% Subfigures & captions
\usepackage{subcaption}
\usepackage{wrapfig}

\DeclarePairedDelimiter\abs{\lvert}{\rvert}
% 
% Source code & highlighting
\usepackage{listings}

% Convenience commands
\newcommand{\mailsubject}{2003 - Biochemie 3}
\newcommand{\maillink}[1]{\href{mailto:#1?subject=\mailsubject}
                               {#1}}

% Should use this command wherever the print date is mentioned.
\newcommand{\printdate}{\today}

% Empty header/footer
\pagestyle{empty}
% Disable header/footer on title page too
\renewcommand*{\titlepagestyle}{empty}

\subject{1611 - Biochemie 3}

\title{Summary}

\author{Michael Senn \maillink{michael.senn@students.unibe.ch} - 16-126-880}

\date{\printdate}

% Needs to be the last command in the preamble, for one reason or
% another. 
\usepackage{hyperref}

\begin{document}
% Space, we need space
% \maketitle
% \vspace*{-20pt}
% \enlargethispage{40pt}

\section{Gene und Chromosome}

\subsection{DNA als Erbsubstanz}

\begin{description}
	\item[Transkription] DNA -> RNA durch RNA Polymerase. In Bakterium in
		Cytoplasma, in Eukaryonten in Zellkern inklusive pre-processing
		(splicing, capping)
	\item[Translation] RNA -> Protein durch Ribosome. In Bakterium und
		Eukaryonten in Cytoplasma.
	\item[DNA \& RNA-bindende Proteine] Schützen, organisieren \&
		regulieren ua durch Konformationsänderung Aspekte des
		Metabolismus. Non-specific binden unabhängig von Sequenz durch
		zB elektrostatische Interaktionen mit Phosphat Gruppen, für zB
		Chromosom Kompaktierung. Specific erkennt spezifisch Sequenz
		durch H-Brücken mit BP, zB Regulation der Gen Expression.
		Kooperativität: Konformationsänderung einer Untereinheit führt
		zu Konfand anderer UE. Stark reguliert.
\end{description}

\subsection{Struktur der DNA}

\begin{description}
	\item[Nukleosid / Nukleotid] (Desoxy)ribose \& Stickstoffbase vs
		Nukleosid \& Phosphatgruppe
	\item[Chemische Transformation von Nukleinsäuren] Spontan (Mutationen)
		zB Deaminierung, enzymatisch reguliert zB Methylierung
	\item[Base stacking] Stapeln der hydrophoben BP führt zu weniger
		Kontakt mit Wasser, fördert Stabilität der DNA Helix.
	\item[DNA Schmelzpunkt Einfluss] GC führt zu höherer Temperatur
		(H-Bond), Ionenkonzentration stabilisiert Phosphat-Gruppen,
		Chaotrope Stoffe destabilisieren H-Bond, pH (de)protoniert
		Gruppen.
	\item[Hairpin / Cruciform / Mirror repeat] Hairpin falls
		Schlaufenbildung aufgrund eines 'inverted repeats' einer
		single-stranded DNA. Cruciform falls Doppelte Schlaufe wegen
		double-stranded. Mirror repeat falls Sequenz mit 'Spiegelachse'
		in der Mitte.
	\item[A / B / Z Form] B Form am stabilsten unter physiologischen
		Bedingungen, \SI{3.4}{\angstrom} per BP, \SI{6}{\degree} BP
		tilt. A in Abwesenheit von Wasser, \SI{2.6}{\angstrom} per BP,
		\SI{20}{\degree} BP tilt. Z Form linksgängig,
		\SI{3.7}{\angstrom} per BP, \SI{7}{\degree} BP tilt, bildet
		sich unter Hochsalzbedingungen mit alternierenden
		Pyrmidin-Purin Sequenzen.
	\item[Supercoiling] Verdichtung zirkulärer DNA Moleküle durch
		verdrehen. L (linking Number) wie oft ein Strang um anderen
		gewunden. Änderung bedingt DNA Schnitt. Negative/positiver
		supercoil bei under/overwinding. Topoisomerasen verändern
		linking number. Typ 1 ohne ATP schneidet ein Strang, Typ 2 mit
		ATP schneidet beide.
\end{description}

\section{Transkription - von der DNA zur RNA}

\section{Post-transkriptionelle RNA Prozessierung}

\section{Protein Synthese}
 
% \section{Moleküle des Lebens}

% \begin{description}
% 	\item[Leben] Weit weg von chem. Gleichgewicht da zB Produkte System
% 		verlassen, Selbstreplikation mit Fehlermöglichkeit (=
% 		Evolution). Braucht Energie.
% \end{description}
% 
% \begin{figure}
% 	\centering
% 	\includegraphics[width=0.8\linewidth]{graphs/ec_quellen.png}
% 	\caption{Classification based on carbon and energy source}
% \end{figure}
% 
% \begin{tabu}{lll}
% 	\toprule
% 	Struktur & Aufbau & Funktion \\
% 	\midrule
% 	Zellwand & Peptidoglycan & Mechanischer Support \\
% 	\bottomrule
% \end{tabu}


\end{document}
