\documentclass[a4paper,twocolumn,usegeometry,english,fontsize=6,DIV=16]{scrartcl}

% Localization options
\usepackage[english]{babel}
\usepackage{blindtext}
\usepackage[T1]{fontenc}
\usepackage[utf8]{inputenc}

% Enhanced verbatim sections. We're mainly interested in
% \verbatiminput though.
\usepackage{verbatim}

% PDF-compatible landscape mode.
% Makes PDF viewers show the page rotated by 90°.
\usepackage{pdflscape}

% Houston, we need to save space.

\usepackage[bottom=2cm,top=2cm]{geometry}

% No vertical space inbetween lists.
\usepackage{enumitem}
\setlist{nosep}

% Remove space between paragraphs.
\setlength{\parskip}{0pt}
\setlength{\parsep}{0pt}
\setlength{\headsep}{0pt}
\setlength{\topskip}{0pt}
\setlength{\topmargin}{0pt}
\setlength{\topsep}{0pt}
\setlength{\partopsep}{0pt}

% Lower space between lines.
% \linespread{0.8}

% Get rid of space above and below section headings
\RedeclareSectionCommand[beforeskip=0pt,afterskip=0pt]{section}
\RedeclareSectionCommand[beforeskip=0pt,afterskip=0pt]{subsection}
\RedeclareSectionCommand[beforeskip=0pt,afterskip=0pt]{subsubsection}

% Advanced tables
\usepackage{tabu}
\usepackage{longtable}
\usepackage{dcolumn}
\newcolumntype{d}[1]{D{.}{\cdot}{#1} }

% Fancy tablerules
\usepackage{booktabs}

% Graphics
\usepackage{graphicx}

% Current time
\usepackage[useregional=numeric]{datetime2}

% Float barriers.
% Automatically add a FloatBarrier to each \section
% \usepackage[section]{placeins}

\usepackage{layout}

% Math tools
\usepackage{mathtools}
% Math symbols
\usepackage{amsmath,amsfonts,amssymb}
\usepackage{amsthm}

% SI units
\usepackage{siunitx}
\DeclareSIUnit\Molar{\textsc{m}}
\DeclareSIUnit\mole{mol}
\DeclareSIUnit\rpm{rpm}
\DeclareSIUnit\cfu{cfu}

% Chemistry
\usepackage{mhchem}

% Subfigures & captions
\usepackage{subcaption}
\usepackage{wrapfig}

\DeclarePairedDelimiter\abs{\lvert}{\rvert}
% 
% Source code & highlighting
\usepackage{listings}

% Convenience commands
\newcommand{\mailsubject}{2003 - Biochemie 3}
\newcommand{\maillink}[1]{\href{mailto:#1?subject=\mailsubject}
                               {#1}}

% Should use this command wherever the print date is mentioned.
\newcommand{\printdate}{\today}

% Empty header/footer
\pagestyle{empty}
% Disable header/footer on title page too
\renewcommand*{\titlepagestyle}{empty}

\subject{1611 - Biochemie 3}

\title{Summary}

\author{Michael Senn \maillink{michael.senn@students.unibe.ch} - 16-126-880}

\date{\printdate}

% Needs to be the last command in the preamble, for one reason or
% another. 
\usepackage{hyperref}

\begin{document}

% Space, we need space
% \maketitle
% \vspace*{-20pt}
% \enlargethispage{40pt}

\section{Gene und Chromosome}

\subsection{DNA als Erbsubstanz und Struktur der DNA}

\begin{description}
	\item[Transkription] DNA -> RNA durch RNA Polymerase. In Bakterium in
		Cytoplasma, in Eukaryonten in Zellkern inklusive pre-processing
		(splicing, capping)
	\item[Translation] RNA -> Protein durch Ribosome. In Bakterium und
		Eukaryonten in Cytoplasma.
	\item[DNA \& RNA-bindende Proteine] Schützen, organisieren \&
		regulieren ua durch Konformationsänderung Aspekte des
		Metabolismus. Non-specific binden unabhängig von Sequenz durch
		zB elektrostatische Interaktionen mit Phosphat Gruppen, für zB
		Chromosom Kompaktierung. Specific erkennt spezifisch Sequenz
		durch H-Brücken mit BP, zB Regulation der Gen Expression.
		Kooperativität: Konformationsänderung einer Untereinheit führt
		zu Konfand anderer UE. Stark reguliert.
	\item[Nukleosid / Nukleotid] (Desoxy)ribose \& Stickstoffbase vs
		Nukleosid \& Phosphatgruppe
	\item[Chemische Transformation von Nukleinsäuren] Spontan (Mutationen)
		zB Deaminierung, enzymatisch reguliert zB Methylierung
	\item[Base stacking] Stapeln der hydrophoben BP führt zu weniger
		Kontakt mit Wasser, fördert Stabilität der DNA Helix.
	\item[DNA Schmelzpunkt Einfluss] GC führt zu höherer Temperatur
		(H-Bond), Ionenkonzentration stabilisiert Phosphat-Gruppen,
		Chaotrope Stoffe destabilisieren H-Bond, pH (de)protoniert
		Gruppen.
	\item[Hairpin / Cruciform / Mirror repeat] Hairpin falls
		Schlaufenbildung aufgrund eines 'inverted repeats' einer
		single-stranded DNA. Cruciform falls Doppelte Schlaufe wegen
		double-stranded. Mirror repeat falls Sequenz mit 'Spiegelachse'
		in der Mitte.
	\item[A / B / Z Form] B Form am stabilsten unter physiologischen
		Bedingungen, \SI{3.4}{\angstrom} per BP, \SI{6}{\degree} BP
		tilt. A in Abwesenheit von Wasser, \SI{2.6}{\angstrom} per BP,
		\SI{20}{\degree} BP tilt. Z Form linksgängig,
		\SI{3.7}{\angstrom} per BP, \SI{7}{\degree} BP tilt, bildet
		sich unter Hochsalzbedingungen mit alternierenden
		Pyrmidin-Purin Sequenzen.
	\item[Supercoiling] Verdichtung zirkulärer DNA Moleküle durch
		verdrehen. L (linking Number) wie oft ein Strang um anderen
		gewunden. Änderung bedingt DNA Schnitt. Negative/positiver
		supercoil bei under/overwinding. Topoisomerasen verändern
		linking number. Typ 1 ohne ATP schneidet ein Strang, Typ 2 mit
		ATP schneidet beide. NB: Eukaryontische Topoisomerasen können
		nur positive supercoil aufheben, keine negative supercoil
		induzieren!
	\item[Nukleotiden] Purin: Adenine, Guanine, Pyrmidin: Uracil, Thymine,
		Cytosine.
\end{description}

\subsection{DNA Schäden \& Reparatur}

\begin{description}
	\item[DNA Schäden] Hydrolyse der Ribose-Basen Bindung (häufig) oder
		Phosphodiesterbindung (selten). Methylierung verschiedener
		Stickstoffe der Basen (mittel), Oxidierung der Basen oder
		Zucker (selten - mittel)
	\item[Mutation] Schaden der vor Replikation nicht behoben wurde. Treibende Kraft der Evolution, vererbbar.
	\item[Ames Test] Inkubation von Bakterien ohne His-Synthese auf Platten
		ohne His. Nur jene die durch Mutationen (spontan oder durch
		Stoff induziert) His syntethisieren können wachsen. Führt zu
		stärkerem Wachstum falls mit mutagenen Stoff in Filterpapier in
		Mitte der Platte behandelt, exklusive 'clear zone' in jenem
		Bereich bei welchem Stoff toxisch / zu viele Mutationen.
	\item[Substitutionsmutation / Punktmutation] Austasch eines BP gegen
		anderes. (Transition: Purin gg Purin / Pyrmidin gg Pyrmidin,
		Transversion: Anders). Aufgrund zweisträngiger DNA sind Point
		mutations nach 1 Generation noch erkennbar, nach 2ter nicht
		mehr.
	\item[Indel] Insertion / Deletion
	\item[Silent / neutral mutation] Ohne Auswirkung auf Genfunktion / AS-Sequenz
	\item[Single nucleotide polymorphism SNP] Kleine Mutationen in Genen
		von Individuuen gleicher Spezies.
	\item[Desaminierung] Ca 100 / Zelle / Tag. Erkennbar \& reparierbar da
		immer Degradation zu BP das nicht in DNA vorkommt.
	\item[Depurinierung] \si{10E-5} / Zelle / Tag. Verlust der Base. Führt
		zu -1 del nach Replikation.
	\item[Oxidative Modifikationen] durch zB ROS. Kann dazu führen dass
		Base mit neuer Base paaren kann, bei falscher Reparatur oder
		bei Replikation damit zu Point Mutation.
	\item[Basenmethylierung] durch alkylierende Substanzen. Kann dazu
		führen dass neue Basen-Paarungen, kann still sein, oder alles
		dazwischen.
	\item[Pyrimidindimere] Durch UV Strahlung. Basen-Basen Bindung führt zu
		Knick in DNA, blockiert Polymerase.
	\item[Rückgratbrucht durch ionisierende Strahlung]
	\item[Indirekte Reparatur] Betroffener Bereich abgebaut und neu
		synthetisiert.
	\item[Direkte Reparatur] Schaden repariert, modifizerte Base
		wiederhergestellt, Monomer gebrochen etc. Bsp Reparatur von
		Thymindimeren durch Photolyase, oder Transfer der Methyl-Gruppe
		alkylierter Basen auf `Opfer'-Molekül durch Methyltransferase.
	\item[Mismatch Reparatur MMR] Erkennt Template-Strang aufgrund
		Methylierung da neuer Strang nicht sofort methyliert. Damit MMR
		auf nicht-methylierten Strang/Strängen. Bildung Loop mit
		Mismatch darin zwischen zwei Methylierten Stellen, ausschneiden
		des Bereichs des Template Stranges, Replikation.
	\item[Base excision repair BER] Erkennen aller Basen die keine
		H-Brücken mehr bilden (Aufgrund zB Depurinierung oder
		Deaminierung). Ausschneiden des leeren Rückgrates, Replikation.
	\item[Nucleotide excision repair NER] Bei zB Pyrimidindimeren
		Ausschneiden eines n-mers des beschädigten Stranges,
		Replikation.
	\item[Doppelstrangbrüche] Während und kurz nach Replikation: Reparatur
		durch Homologe Rekombination (Ausrichtung an Schwesterstrang,
		dan erneute Replikation). Ansonsten: Nonhomologous end joining,
		Abschneiden des Überhanges, dann zusammenfügen.
	\item[SOS Response] DNA Schaden -> Inaktiviert LexA -> Expression SOS
		Gene (Mehr und weitere DNA Reparatur Gene, keine Zellteilung).
		Inaktivierung LexA durch RecA, welches a) DNA Schäden bindet
		und b) in gebundenem Zustand LexA spaltet.
\end{description}

\subsection{Chromatinstruktur}
\begin{description}
	\item[Chromatin] Komplex aus DNA um Histonen (= Nucleosom) und anderen
		Proteinen, Erlaubt Konzentration der DNA bis zu Faktor 700.
		Histone sehr basisch (Lys, Arg) / positiv, stark konserviert.
		Euchromatin locker, Heterochromatin gewunden (Histone
		kondensiert).  Acetylierung führt zu Öffnung, Methylierung zu
		Kondensation.
	\item[Histon] Protein Komplex, 8 UE, H2A, H2B, H3, H4, 146bp DNA.
		Helferproteine für Assembly des Oktamers. Interaktion mit DNA
		hauptsächlich mit Rückgrat und Minor Groove, damit
		\SI{75}{\percent} der DNA Oberfläche Zugänglich für zB
		Regulation. H1 `hält' DNA um Histon fest. Mit H1 200 bp.
	\item[Histonenden] Unstrukturiert / loose. Tragen zur höheren
		Organisation (zB \SI{30}{\nm} Faser) bei.
	\item[Unterwundene eukaryontische DNA] Binden DNA an Histon Core
		induziert negative Supercoil, damit positiven Supercoil in der
		Nähe. Dieser durch Topoisomerase aufgehoben, damit netto
		unterwunden.
	\item[Chromatin Immunoprecipitation (ChIP)] Isolierung Chromatin aus
		Zelle, Schneiden mit Nuclease (ca 500bp Fragmente) Bindung und
		Isolierung spezifischer Histon-Modifikationen mit Antikörpern,
		Extrahierung \& Sequenzierung DNA. Dadurch Erkenntniss in
		welchem DNA Bereich welche Histon-Modifikationen vorkommen.
	\item[SMC Proteine] Cohesin verhindert Dissoziation von
		Schwester-Chromatiden, Condensin helfen Chromatide kondensiert
		zu halten. Cohesin  wird in Anaphase entfernt damit Aufteilung
		auf Tochterzellen, nach Chromatinseparation dissazoziiert
		Condensin damit dekondensation der DNA.
\end{description}

\subsection{DNA Replikation}
\begin{description}
	\item[Semi-konservativ] Bei Replikation werden Stränge des
		Doppelstranges getrennt und b) bleiben Stränge intern als
		Einheit.
	\item[5' -> 3'] Des neuen Stranges, dh 3' -> 5' des Template Stranges.
		Grund: Neue Phosphat-Gruppe wird an alte Hydroxy-Gruppe
		angehängt. Benötigt RNA Primer.
	\item[Okazaki Fragmente] 1k-2k nt in Prokaryonten, 150-200 nt in
		Eukaryonten.
	\item[Replikation] Bidirektional ab ORI, Replication Fork bei Trennung
		des Doppelstranges. Benötigt RNA Primer und \ce{Mg2+} Ionen.
\end{description}

\section{Transkription - von der DNA zur RNA}
Test

\section{Post-transkriptionelle RNA Prozessierung}
Test

\section{Protein Synthese}
Test

\end{document}
